%! Author = chiel
%! Date = 10-02-20

% Preamble
\documentclass[11pt]{article}

\title{Implementatieplan Week 2 Lokalisatie Stap 4: Ogen en Neus}
\author{Chiel Douwes}

% Document
\begin{document}

    \maketitle


    \section{Doel}\label{sec:doel}
    % Geef aan wat het doel van de implementatie is.
    De implementatie heeft als doel om de ogen en de neus van de persoon te kunnen lokaliseren.


    \section{Methoden}\label{sec:methoden}
    % Je geeft hier aan welke methoden er zijn, wat de verschillende tussen de methodes zijn.
    Als methoden kunnen er of een meetkundige methode of een vision processing methode toegepast
    worden, het nadeel van een meetkundige methode is dat de echte locaties per persoon zullen
    verschillen, wat niet gereflecteerd wordt.
    In de vision processing methode wordt handmatig gezocht naar de ogen en de neus, en dit zal
    dus ook leiden tot betere resultaten.


    \section{Keuze}\label{sec:keuze}
    % Je geeft een onderbouwing over waarom een bepaalde methode is gekozen,
    % en/of waarom bepaalde settings zijn gebruikt.
    De keuze is gemaakt om met een vision processing methode de localisatie uit te voeren om de
    in de Methoden genoemde redenen, naast dat het een beter voorbeeld is naar het invullen van
    de leerdoelen van de cursus Vision.


    \section{Implementatie}\label{sec:implementatie}
    % Je geeft aan hoe deze keuze is geimplementeerd in de code
    De implementatie zoekt eerst naar de zijkanten van de neus door vanaf de wangen in rijen naar
    het midden van het gezicht toe te scannen, waarna uit deze lijst van punten de twee punten
    worden genomen aan bijde zijden die het verste zijn verwijdert van de middenlijn van het
    gezicht.
    Na deze scan worden de ogen op gespoord door eerst een verticaal histogram te maken rondom het
    gebied van de ogen, en in dit histogram wordt de eerste reeks die boven een zekere threshold
    zit aangemerkt als de verticale positie van de ogen.
    Voor de horizontale positie word juist geen gebruik gemaakt van een histogram omdat de
    neusbrug een obstakel kan zijn in het opsporen van de ogen in het verticale domein;
    inplaats daarvan wordt er gebruik gemaakt van een meetkundige methode voor de horizontale
    positie omdat dit een hoog genoege precizie op levert


    \section{Evaluatie}\label{sec:evaluatie}
    % Je geeft aan welke experimenten er gedaan zullen worden om de implementatie
    % te testen en te ‘bewijzen’ dat de implementatie daadwerkelijk correct werkt.
    % Dit geeft direct informatie over de meetrapporten die er zullen worden gemaakt.
    De code zal getest worden door het te integreren in het complete vision systeem, en
    vervolgens te evalueren of de waardes die daaruit resulteren binnen goede grenzen vallen.

\end{document}
