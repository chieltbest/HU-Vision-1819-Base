%! Author = chiel
%! Date = 10-02-20

% Preamble
\documentclass[11pt]{article}

\title{Implementatieplan Practicum Week 1}
\author{Chiel Douwes}

% Document
\begin{document}

    \maketitle


    \section{Doel}\label{sec:doel}
    % Geef aan wat het doel van de implementatie is.
    De implementatie van de Image-klasse moet de volgende dingen bieden:
    \begin{itemize}
        \item Een dynamische opslag van het type van de pixels die van groote veranderd kan worden
        \item Methodes om te indexen in de pixel array en om deze te lezen en schrijven
        \item Een interface om deze functionaliteiten aan te bieden aan de meegeleverde
        RGBImageStudent en IntensityImageStudent classes
    \end{itemize}
    Er hoeft in de implementatie geen garanties geleverd te worden over onder meer:
    \begin{itemize}
        \item De run-time complexiteit
        \item De access time constraints
        \item Plaatsing van de opslag van het object
        \item De memory safety
        \item De memory contiguity
        \item De waarde of het runtime gedrag van uninitialized values/pixels
        \item De staat van het object na een out of memory condition tijdens een set operatie
        \item De hoeveelheid geheuhengebruik in het object of tijdens het aanroepen van een van
        de functies
        \item De exception-safety van de functies in het object, behalve onder bepaalde precondities
    \end{itemize}
    De eisen die wel gesteld worden zijn onder meer:
    \begin{itemize}
        \item Get-operaties moeten geen size effects hebben in de pixel data
        \item De pixel data dat gezet in na het opzetten van het object moet in redelijke mate
        onveranderlijk zijn tijdens de lifetime van het object
    \end{itemize}


    \section{Methoden}\label{sec:methoden}
    % Je geeft hier aan welke methoden er zijn, wat de verschillende tussen de methodes zijn.
    Er zijn enkele methoden die gebruikt kunnen worden:
    \begin{itemize}
        \item Fixed-size array: De simpelste implementatie, met een vaste hoogte en breedte
        \item Sparse array: Een array die stukken kan bevatten die niet gealloceerd zijn
        \item Dynamic array: Een array die naar vraag groter of kleiner kan worden gemaakt
    \end{itemize}
    Voor de fixed-size array kan de storage op verschillende plekken geplaatst worden afhankelijk
    van de grootte van het array en de geheugen opties in de target architectuur;
    zo kan het bijvoorbeeld op de stack worden geplaatst in het geval van grootes van minder dan
    een aantal kilobytes, of op de heap voor grotere images.
    Dit geld ook voor de sparse array en dynamic array, maar deze zijn iets flexibeler in het
    plaatsen van het geheugen en in de allocatie van geheugen tijdens de set en get methodes.


    \section{Keuze}\label{sec:keuze}
    % Je geeft een onderbouwing over waarom een bepaalde methode is gekozen,
    % en/of waarom bepaalde settings zijn gebruikt.
    De keuze wordt gemaakt om een fixed-size array te implementeren, mits dit de snelste array
    access oplevert en ook een correcte implementatie gemakkelijk te verifiëren is.
    Naast deze voordelen is er in het fixed-size array ook geen mogelijkheid tot momory errors of
    allocation failures tijdens de set-methodes sinds deze nooit extra geheugen zal hoeven te
    alloceren.


    \section{Implementatie}\label{sec:implementatie}
    % Je geeft aan hoe deze keuze is geimplementeerd in de code
    Het fixed-size array is geïmplementeerd door een type te maken met vaste hoogte en breedte
    parameters, zodat de access instructies in compile time geoptimalizeerd kunnen worden
    doormiddel van SIMD. Omdat de implementatie ook functionaliteit moet leveren om de grootte
    van de image te veranderen is er ook de optie om de klasse een dynamische grootte te geven
    door de parameters naar 0 te zetten.
    Het type heeft als een van de variabelen de data van de image.
    Er zal een functie gegeven worden die kan indexen in de array, en deze methode zal gebruikt
    worden om de get implementatie te maken van de RGB/IntensityImageStudent, maar sinds de
    methode een reference geeft kan het ook gebruikt worden om de set methode te implementeren.


    \section{Evaluatie}\label{sec:evaluatie}
    % Je geeft aan welke experimenten er gedaan zullen worden om de implementatie
    % te testen en te ‘bewijzen’ dat de implementatie daadwerkelijk correct werkt.
    % Dit geeft direct informatie over de meetrapporten die er zullen worden gemaakt.
    De experimenten die uitgevoerd kunnen worden zijn onder anderen tests om de correctness van
    de implementatie te testen, zoals out of bounds checks en andere edge cases, en er kunnen
    tests gemaakt worden om de performance van de pixel update methodes te testen.
    Voor de evaluatie van de performance van de image klasse moeten er meerdere patronen getest
    worden sinds compiler optimalisatie en cacheline grootte een impact kunnen hebben in hoe snel
    de test wordt uitgevoerd.
    Om deze reden is het ook een goed idee om meerdere grootes images te testen in combinatie met
    verschillende access patronen.
    Om de correctness van de image klasse aan te tonen is het ook mogelijk om een integration
    test te doen met het image processing programma dat mee geleverd is voor de cursus Vision.


\end{document}
