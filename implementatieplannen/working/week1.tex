%! Author = chiel
%! Date = 10-02-20

% Preamble
\documentclass[11pt]{article}

\title{Implementatieplan Practicum Week 1}
\author{Chiel Douwes}

% Document
\begin{document}

    \maketitle


    \section{Doel}\label{sec:doel}
    % Geef aan wat het doel van de implementatie is.
    De geimplementeerde klasse moet een interface geven die het mogelijk maakt om in een image te
    lezen en te schrijven.


    \section{Methoden}\label{sec:methoden}
    % Je geeft hier aan welke methoden er zijn, wat de verschillende tussen de methodes zijn.
    Er zijn enkele methoden die gebruikt kunnen worden:
    \begin{itemize}
        \item Fixed-size array: De simpelste implementatie, met een vaste hoogte en breedte.
        \item Sparse array: Een array die stukken kan bevatten die niet gealloceerd zijn.
        \item Dynamic array: Een array die naar vraag groter of klijner kan worden gemaakt.
    \end{itemize}


    \section{Keuze}\label{sec:keuze}
    % Je geeft een onderbouwing over waarom een bepaalde methode is gekozen,
    % en/of waarom bepaalde settings zijn gebruikt.
    De keuze wordt gemaakt om een fixed-size array te implementeren, mits dit de snelste array
    access oplevert en ook een correcte implementatie gemakkelijk te verifiëren is.


    \section{Implementatie}\label{sec:implementatie}
    % Je geeft aan hoe deze keuze is geimplementeerd in de code
    Het fixed-size array is geïmplementeerd door een type te maken met vaste hoogte en breedte
    parameters, zodat de access instructies in compile time geoptimalizeerd kunnen worden
    doormiddel van SIMD. Het type heeft als een van de variabelen de data van de image. Er zullen
    enige functies zijn die bepaalde pixels kunnen zetten of getten, met zo nodig boundary checks
    for out of bounds condities.


    \section{Evaluatie}\label{sec:evaluatie}
    % Je geeft aan welke experimenten er gedaan zullen worden om de implementatie
    % te testen en te ‘bewijzen’ dat de implementatie daadwerkelijk correct werkt.
    % Dit geeft direct informatie over de meetrapporten die er zullen worden gemaakt.
    De experimenten die uitgevoerd kunnen worden zijn onder anderen tests om de correctness van
    de implementatie te testen, zoals out of bounds checks en andere edge cases, en er kunnen
    tests gemaakt worden om de performance van de pixel update methodes te testen.


\end{document}
