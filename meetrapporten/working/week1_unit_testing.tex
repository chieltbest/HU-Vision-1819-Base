%! Author = chiel
%! Date = 10-02-20

% Preamble
\documentclass[11pt]{article}

\title{Meetrapport Unit Testing}
\author{Chiel Douwes}

% Document
\begin{document}

    \maketitle


    \section{Doel}\label{sec:doel}
    % Geef aan wat het doel van het experiment is, bijvoorbeeld in de vorm van een te controleren
    % hypothese.
    Er worden unit tests van de code gemaakt om te er zeker van te zijn dat de code correct
    uitvoert in de geteste situaties.


    \section{Hypothese}\label{sec:hypothese}
    % Voordat je aan de proef begint stel je een hypothese op;
    % wat verwacht je dat het antwoord zal zijn op je onderzoeksvraag?
    Het is verwacht dat de code en tests allemaal succesvol gecompileerd kunnen worden en ook de
    correcte waardes produceren.


    \section{Werkwijze}\label{sec:werkwijze}
    % Geef een korte beschrijving van het experiment.
    % (Het overschrijven van de practicumhandleiding is niet nodig.) Maak indien nodig een
    % tekening van de proefopstelling, waarin grootheden kunnen worden aangegeven.
    Er worden functies gemaakt die gebruik maken van de mehtodes in de classes die getest worden,
    om zo te evalueren of deze functies goed werken.


    \section{Resultaten}\label{sec:resultaten}
    % Geef de meetresultaten overzichtelijk weer in de vorm van een tabel en/of diagram.
    Alle code compileert en gedraagt zich als verwacht.


    \section{Verwerking}\label{sec:verwerking}
    % Laat zien hoe je de meetresultaten verwerkt om een conclusie te kunnen trekken.
    % Het is niet nodig om alle berekeningen op te schrijven, als je bijvoorbeeld maar laat zien
    % welke formule(s) je gebruikt voor het verwerken van de meetresultaten en daar zo nodig één
    % voorbeeldberekening aan toevoegt.
    Sinds alle code goed werkt is de kans dat het in andere code ook goed werkt groot.


    \section{Conclusie}\label{sec:conclusie}
    % Geef aan welke conclusie kan worden getrokken uit de verwerking van de meetresultaten.
    De code is in goede staat en voldoet aan het ontwerp, dus de implementatie is naar weten
    correct.


    \section{Evaluatie}\label{sec:evaluatie}
    % Leg een verband tussen de getrokken conclusie en het doel van het experiment (en de hypothese).
    % Ga daarbij ook in op bijvoorbeeld de meetonzekerheid als gevolg van de gebruikte
    % meetmethoden of eventuele meetfouten.
    Het doel van het meetrapport is volbracht, sinds alle tests werken en in goede staat zijn.
    Inbegrepen in deze tests is het maken van een image, pixels zetten en getten, en conversie
    naar intensity.

\end{document}
