%! Author = chiel
%! Date = 10-02-20

% Preamble
\documentclass[11pt]{article}

\title{Meetrapport Week 1 Benchmarks}
\author{Chiel Douwes}

% Document
\begin{document}

    \maketitle


    \section{Doel}\label{sec:doel}
    % Geef aan wat het doel van het experiment is, bijvoorbeeld in de vorm van een te controleren
    % hypothese.
    De code wordt gebenchmarkt om te verifiëren dat het binnen goede tijd een resultaat kan geven.


    \section{Hypothese}\label{sec:hypothese}
    % Voordat je aan de proef begint stel je een hypothese op;
    % wat verwacht je dat het antwoord zal zijn op je onderzoeksvraag?
    De benchmarks moeten binnen een redelijke tijd uit te voeren zijn nadat deze gecompileerd zijn.


    \section{Werkwijze}\label{sec:werkwijze}
    % Geef een korte beschrijving van het experiment.
    % (Het overschrijven van de practicumhandleiding is niet nodig.) Maak indien nodig een
    % tekening van de proefopstelling, waarin grootheden kunnen worden aangegeven.
    Voor de benchmarks wordt een bestand aangemaakt met functies, en dit bestand wordt vervolgens
    door clang gecompileerd met een optimalisatie level van -O3. Door het gebruik van std::chrono
    kunnen de tijden met hoge precisie op genomen worden.
    \linebreak
    In de tests van deze benchmark zal een Image van 1000x1000 pixels aangemaakt worden, daarna
    gevult met een patroon en vervolgens geconverteerd naar grayscale.


    \section{Resultaten}\label{sec:resultaten}
    % Geef de meetresultaten overzichtelijk weer in de vorm van een tabel en/of diagram.
    \begin{itemize}
        \item Image constructor: $\approx$ 1.5ms
        \item Vullen van Image: $\approx$ 0.8ms
        \item Converteren van Image naar grayscale: $\approx$ 6.5ms
    \end{itemize}
    Voor alle metingen wordt van een aantal metingen de minimum tijd genomen, sinds dit het de
    meest betrouwbare tijd is zonder invloed van de linux scheduler.


    \section{Verwerking}\label{sec:verwerking}
    % Laat zien hoe je de meetresultaten verwerkt om een conclusie te kunnen trekken.
    % Het is niet nodig om alle berekeningen op te schrijven, als je bijvoorbeeld maar laat zien
    % welke formule(s) je gebruikt voor het verwerken van de meetresultaten en daar zo nodig één
    % voorbeeldberekening aan toevoegt.
    Alle tijden zijn binnen bruikbare waardes, wat het mogelijk maakt om of in realtime of om
    grotere berekeningen te doen op de images.


    \section{Conclusie}\label{sec:conclusie}
    % Geef aan welke conclusie kan worden getrokken uit de verwerking van de meetresultaten.
    De code voldoet aan de eisen die gesteld zijn aan de performance, en kan dus ook in andere
    applicaties toe gepast worden.


    \section{Evaluatie}\label{sec:evaluatie}
    % Leg een verband tussen de getrokken conclusie en het doel van het experiment (en de hypothese).
    % Ga daarbij ook in op bijvoorbeeld de meetonzekerheid als gevolg van de gebruikte
    % meetmethoden of eventuele meetfouten.
    De code is goed getest en de meetwaardes zijn met een marge van ongeveer \%10 goed toepasbaar
    in vergelijkbare situaties in komende code.

\end{document}
