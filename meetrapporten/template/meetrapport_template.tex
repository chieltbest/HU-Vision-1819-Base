%! Author = chiel
%! Date = 10-02-20

% Preamble
\documentclass[11pt]{article}

\title{Meetrapport}
\author{Chiel Douwes}

% Document
\begin{document}

    \maketitle


    \section{Doel}\label{sec:doel}
    % Geef aan wat het doel van het experiment is, bijvoorbeeld in de vorm van een te controleren
    % hypothese.


    \section{Hypothese}\label{sec:hypothese}
    % Voordat je aan de proef begint stel je een hypothese op;
    % wat verwacht je dat het antwoord zal zijn op je onderzoeksvraag?


    \section{Werkwijze}\label{sec:werkwijze}
    % Geef een korte beschrijving van het experiment.
    % (Het overschrijven van de practicumhandleiding is niet nodig.) Maak indien nodig een
    % tekening van de proefopstelling, waarin grootheden kunnen worden aangegeven.


    \section{Resultaten}\label{sec:resultaten}
    % Geef de meetresultaten overzichtelijk weer in de vorm van een tabel en/of diagram.


    \section{Verwerking}\label{sec:verwerking}
    % Laat zien hoe je de meetresultaten verwerkt om een conclusie te kunnen trekken.
    % Het is niet nodig om alle berekeningen op te schrijven, als je bijvoorbeeld maar laat zien
    % welke formule(s) je gebruikt voor het verwerken van de meetresultaten en daar zo nodig één
    % voorbeeldberekening aan toevoegt.


    \section{Conclusie}\label{sec:conclusie}
    % Geef aan welke conclusie kan worden getrokken uit de verwerking van de meetresultaten.


    \section{Evaluatie}\label{sec:evaluatie}
    % Leg een verband tussen de getrokken conclusie en het doel van het experiment (en de hypothese).
    % Ga daarbij ook in op bijvoorbeeld de meetonzekerheid als gevolg van de gebruikte
    % meetmethoden of eventuele meetfouten.

\end{document}
